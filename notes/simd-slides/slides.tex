%----------------------------------------------------------------------------------------
%	PACKAGES AND THEMES
%----------------------------------------------------------------------------------------
\documentclass[aspectratio=169,xcolor=dvipsnames]{beamer}
\usepackage[T1]{fontenc}
\usepackage[utf8]{inputenc}
\usepackage[english]{babel}
\usetheme{SimplePlus}

\usepackage{hyperref}
\usepackage{graphicx} % Allows including images
\usepackage{booktabs} % Allows the use of \toprule, \midrule and \bottomrule in tables

\usepackage{minted}
\usepackage{amsmath}
\usepackage[table]{xcolor}

%----------------------------------------------------------------------------------------
%	TITLE PAGE
%----------------------------------------------------------------------------------------

\title[short title]{Measureing Usage and Efficiency of SIMD Instructions} % The short title appears at the bottom of every slide, the full title is only on the title page
\author[Sebastian] {Sebastian Dawid}
\date{\today} % Date, can be changed to a custom date


%----------------------------------------------------------------------------------------
%	PRESENTATION SLIDES
%----------------------------------------------------------------------------------------

\begin{document}

\begin{frame}
    % Print the title page as the first slide
    \titlepage
\end{frame}

\begin{frame}{Overview}
    % Throughout your presentation, if you choose to use \section{} and \subsection{} commands, these will automatically be printed on this slide as an overview of your presentation
    \tableofcontents
\end{frame}

%------------------------------------------------
\section{Metrics}
%------------------------------------------------

\begin{frame}{Vector Metrics}
    \begin{itemize}
        \item Waste: Number of instructions executed less than the theoretical maximum. Here I assume that the maximum number of vector instructions per cycle is 2\footnote{A lot of instructions have a throughput(CPI) of 0.5}.
            \begin{equation*}
                \text{waste}(c, v) = 2c - v
            \end{equation*}
        \item Efficiency: Ratio of executed vector instructions to the theoretical maximum.
            \begin{equation*}
                \text{efficiency}(c, v) = \frac{v}{2c}
            \end{equation*}
        \item VPC: Vecror instructions per cycle
            \begin{equation*}
                \text{vpc}(c, v) = \frac{v}{c}
            \end{equation*}
    \end{itemize}
\end{frame}

\begin{frame}{Cache Metric}
    Through the ratio of cache misses to cache accesses we can ensure that we are not observing cache effects:
    \begin{equation*}
        \text{cmpca}(cm, ca) = \frac{cm}{ca}
    \end{equation*}
\end{frame}

\begin{frame}{Hardware Counters}
    To calculate these metrics we need to monitor various hardware events\footnote{Events are captured via PAPI (Performance Application Programming Interface)}:
    \begin{itemize}
        \item \mintinline{c}{PAPI_VEC_INS} (vector instructions executed)
        \item \mintinline{c}{PAPI_TOT_CYC} (total cycles executed)
        \item \mintinline{c}{PAPI_L1_DCM} (L1 data cache misses)
        \item \mintinline{c}{PAPI_L1_DCA} (L1 data cache accesses)
    \end{itemize}
\end{frame}

\section{Example}

\begin{frame}{Program}
    \begin{itemize}
        \item Summing up $\sim128$KiB (32 Pages of 4KiB each) of double precision floating point data repeated 1024 times.
        \item Use one sum function with loop carry dependency and one without.
        \item Store data in circular buffer to save physical RAM.
    \end{itemize}
\end{frame}

\begin{frame}{Measurements}
    \begin{table}
        \centering
        \rowcolors{1}{white}{lightgray}
        \begin{tabular}{|c|c|c|}
            \hline
            Metric              & With Dependency & Without Dependency\\\hline
            $\text{vpc}$        & $0.332$ & $0.492$ \\
            $\text{waste}$      & $1.52 \cdot 10^{10}$ ($62.2\%$) & $9.2 \cdot 10^9$ ($37.7\%$) \\
            $\text{efficiency}$ & $16.62\%$ & $24.58\%$ \\
            CPUTIME             & $1590$ ms ($55.8\%$) & $1040$ ms ($36.3\%$) \\\hline
        \end{tabular}
    \end{table}
\end{frame}

\end{document}
