\documentclass[a4paper, 11pt]{memoir}

\usepackage[T1]{fontenc}
\usepackage[utf8]{inputenc}

\usepackage[english]{babel}

\usepackage{lipsum}
\usepackage[table]{xcolor}

\input{smart-thesis/style}
\input{smart-thesis/common-packages}
\input{smart-thesis/common-macros}

\addbibresource{main.bib}

\makeglossaries
\newglossaryentry{transmittance}{
    name=Transmittance,
    description={The transmittance $T \in [0,1]$ is the fraction of light passing straight between two points}
}

\newglossaryentry{radiance}{
    name=Radiance,
    description={The radiance $L$ is the fraction of reflected light}
}

% TODO: Do I need this?
\newglossaryentry{pinhole_camera}{
    name=Pinhole Camera,
    description={Camera model where a given space is projected onto a plane through a point}
}

\newglossaryentry{lambda}{
    name=Lambda,
    description={A lambda function is a function that is not bound to a symbol at compile. Instead it is bound to a variable at runtime or passed directly to a higher order function as an argument}
}

\newglossaryentry{vulkan}{
    name=Vulkan,
    description={A low-level graphics API and standard developed by the Khronos Group}
}

\newglossaryentry{albedo}{
    name=Albedo,
    description={The fraction of light a body reflects diffusely. Value in $[0, 1]$ given separately for each color channel, where $0$ denotes no reflection and $1$ denotes full reflection}
}

\newglossaryentry{simd}{
    name=SIMD,
    description={SIMD (Single Instruction Multiple Data) is a form of parallelism that performs a single operation on multiple pieces of data in parallel on a single CPU}
}

\newglossaryentry{ndc}{
    name=Normalized Device Coordinates,
    description={A coordinate space normalized to $[-1, 1] \times [-1, 1]$}
}

\newglossaryentry{pipelining}{
    name=Pipelining,
    description={Modern CPUs usually have multiple pipeline stages. This means that a new instruction can be submitted to a pipelined unit before the result of the previous instruction is available}
}

\newglossaryentry{latency}{
    name=Latency,
    description={The number of CPU cycles it takes for an instruction's result to be available}
}

\newglossaryentry{throughput}{
    name=Throughput,
    description={The number of CPU cycles it takes, after an instruction is started, until another instruction can be executed}
}

\newglossaryentry{isotropic}{
    name=Isotropic Gaussian,
    description={An isotropic Gaussian has a scalar value as its standard deviation and is affected by it equally in all dimensions}
}


\thesistype{Bachelor Thesis}
\discipline{Computer Science}
\title{Parallel Gaussian Raytracing on the CPU}
\author{Sebastian Dawid}
\institution{Bielefeld University,Technical Faculty,Visual AI for Extended Reality Group}
\supervisors{Prof.\@~Dr.\@~Helge Rhodin,Prof.\@~Dr.-Ing.\@~Ralf M\"oller}

\newcommand*{\erf}{\text{erf}}

\makepagestyle{abs}

% Display the page number in the footer.
\makeevenfoot{abs}{\thepage}{}{}
\makeoddfoot{abs}{}{}{\thepage}

\begin{document}
    \frontmatter
    \smarttitle
    \newpage
    \tableofcontents*

    \clearpage
    \thispagestyle{abs}
    \abstractintoc
    \begin{abstract}
        \lipsum[1]
    \end{abstract}

    \mainmatter
    \chapter{Introduction}

    \section{Gaussian Ray Tracing}
    In their 2015 Paper \citetitle{Rhodin:2015} \cite{Rhodin:2015} Rhodin \etal propose a volumetric image formation
    model based on a parametric density representation $D(\mathbf{x})$ given as the sum of scaled isotropic Guassians
    $\mathcal{G} = \{ G_q \}_q$. The density $D$ is then given as:
    \begin{align}
        D(\mathbf{x}) = \sum_{G_q \in \mathcal{G}} G_q(\mathbf{x})
        \label{eq:density}\\
        G_q(\mathbf{x}) = c_q \cdot \exp{\left( - \frac{\Vert\mathbf{x} - \mu_q\Vert_2^2}{2\sigma_q^2} \right)}
        \label{eq:gaussian}
    \end{align}
    where $c_q$ describes the magnitude, $\mu_q$ the center and $\sigma_q$ the standard deviation of the Gaussian $G_q$.
    Additionally an \gls{albedo} attribute $\mathbf{a}_q$ is defined for each Gaussian to denote its color. This leads
    to the scene representation $\gamma = \{ c_q, \mu_q, \sigma_q, \mathbf{a}_q \}$. The $\gamma$ will be omitted from
    $G_q(\mathbf{x})$ for readability and since the parameters are given implicitly via $q$.

    To get the final color of a pixel we need to determine the amount of light that reaches the camera from each point
    along a ray. For this purpose we determine the \gls{transmittance} $T$ of a point at distance $s$ along a ray from
    a camera position $\mathbf{o}$ in direction $\mathbf{n}$ as:
    \begin{equation}
        T(\mathbf{o}, \mathbf{n}, s, \gamma) = \exp{\left( - \int_0^s D(\mathbf{o} + t\mathbf{n}) dt \right)}
        \label{eq:transmittance}
    \end{equation}

    For the Gaussian density representation the density along a ray $\mathbf{x} = \mathbf{o} + s\mathbf{n}$ through a
    sum of 3D Gaussians is a sum of 1D Gaussians, where the 1D Gaussians are given by inserting the ray into the
    3D Gaussians $G_q$ (\refeq{eq:gaussian}). This gives us the form
    $\bar{c} \exp{\left( - \frac{(x - \bar{\mu})^2}{2\bar{\sigma}^2} \right)}$ for the 1D scaled Gaussians, with
    $\bar{\mu} = (\mu - \mathbf{o})^T\mathbf{n}$, $\bar{\sigma} = \sigma$ and
    $\bar{c} = c \cdot \exp{\left( - \frac{(\mu - \mathbf{o})^T(\mu - \mathbf{o}) - \bar{\mu}^2}{2\bar{\sigma}^2} \right)}$.

    The \gls{transmittance} can be expressed analytically using the error function
    $\erf{(x)} = \frac{2}{\sqrt{\pi}}\int_0^s \exp{(-t^2)} dt$ and gaussian form of the density, as follows:
    \begin{equation}
        \begin{aligned}
            T(\mathbf{o}, \mathbf{n}, s, \gamma) &= \exp{\left( -\int_0^s
                \sum_q G_q(\mathbf{o} + t\mathbf{n} ) dt \right)}\\
            &= \exp{\left( \sum_q \frac{\bar{\sigma}_q \bar{c}_q}{\sqrt{\frac{2}{\pi}}}
            \left( \erf{\left( \frac{-\bar{\mu}_q}{\sqrt{2}\bar{\sigma}_q} \right)}
            - \erf{\left( \frac{s - \bar{\mu}_q}{\sqrt{2}\bar{\sigma}_q} \right)} \right) \right)}\\
        \end{aligned}
        \label{eq:transmittance_approx}
    \end{equation}

    
    Assuming the elements in the scene emit an equal amount of \gls{radiance}, a ray is shot through each pixel of a virtual
    \gls{pinhole_camera}. The \gls{radiance} can be computed as the product of \gls{transmittance} $T$ (\refeq{eq:transmittance}),
    density $D$ (\refeq{eq:density}), \gls{albedo} $\mathbf{a}$ and the ambient \gls{radiance} $L_e$ integrated along a ray
    $\mathbf{x} = \mathbf{o} + s\mathbf{n}$. We assume the ambient \gls{radiance} to be fixed as $L_e = 1$. As such we
    may disregard it in the following equation:
    \begin{equation}
        L(\mathbf{o}, \mathbf{n}, \gamma) = \int_0^\infty T(\mathbf{o}, \mathbf{n}, s, \gamma)
            \sum_q G_q(\mathbf{o} + s\mathbf{n})\mathbf{a}_q ds
    \end{equation}
    
    This integral may be approximated with sufficient accuracy by sampling around the mean of each Gaussian $G_q$
    a compact interval $S_q = \{ \bar{\mu}_q + k\lambda_q | k \in K \subset \Z \}$. For our purposes it is sufficient
    to choose $\lambda_q \sim \bar{\sigma}_q$ as the step length:
    \begin{equation}
        \hat{L}(\mathbf{o}, \mathbf{n}, \gamma) = \sum_q \mathbf{a}_q \sum_{s \in S_q}
            \lambda_q T(\mathbf{o}, \mathbf{n}, s, \gamma)G_q(\mathbf{o} + s\mathbf{n})
        \label{eq:radiance}
    \end{equation}

    Rhodin \etal describe that local sampling with $\lambda_q = \bar{\sigma}_q$ and
    $K = \{ -4, -3, \dots, 0 \}$ delivers a good enough approximation.

    \section{Gaussian Splatting}
    \cite{kerbl3Dgaussians}
    \cite{volume_splatting}

    \chapter{Optimizations}
    \section{SIMD}
    \subsection{Notation}

    To properly rewrite the method as it is laid out in \cite{Rhodin:2015} it is
    necessary to introduce some notation to work with SIMD vectors. This is
    defined in Table~\ref{tab:notation}:
    \begin{table}[h]
        \centering
        \rowcolors{1}{white}{lightgray}
        \begin{tabular}{|c|c|}
            \hline
            Notation & Definition \\
            \hline
            $x^W$ & SIMD vector containing $W$ elements\\
            $\langle x \rangle^W$ & $x$ broadcast to a SIMD vector of $W$ elements\\
            $\odot$ & elementwise multiplication\\
            $f^W$ & function that produces a SIMD vector containing $W$ elements\\\hline
        \end{tabular}
        \caption{SIMD Notation}
        \label{tab:notation}
    \end{table}
    \subsection{Approaches}
    \section{Tiling}
    
    \chapter{Approximations}
    \section{Exponential Function}
    \subsection{Cubic Spline Interpolation}
    \subsection{Bit Hack}
    \cite{fast_exp}
    
    \section{Error Function}
    \subsection{Cubic Spline Interpolation}
    \subsection{Polynomial Approximation}
    \cite{AbraSteg72}
    
    \chapter{Implementation}
    \cite{own_moeller_16_2}
    
    \chapter{Experiments}
    \cite{hpc_toolkit}
    \section{Approximations}
    \section{Transmittance}
    \section{Full Render}
    
    \chapter{Results and Discussion}

    \chapter{Conclusion and Future Work}

    \backmatter
    \printglossaries
    \printbibliography[heading=bibintoc]
\end{document}
