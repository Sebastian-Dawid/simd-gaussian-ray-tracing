\documentclass[a4paper, 11pt]{memoir}

\usepackage[T1]{fontenc}
\usepackage[utf8]{inputenc}

\usepackage[english]{babel}

%
% Smart Thesis LaTeX template
%
% [To underline the amateurish flavour of this template, let's start with some
% nifty ASCII-Art (http://www.network-science.de/ascii/)...]
%
%      #######
%    /       ###
%   /         ##                                          #
%   ##        #                                          ##
%    ###                                                 ##
%   ## ###      ### /### /###     /###   ###  /###     ########
%    ### ###     ##/ ###/ /##  / / ###  / ###/ #### / ########
%      ### ###    ##  ###/ ###/ /   ###/   ##   ###/     ##
%        ### /##  ##   ##   ## ##    ##    ##            ##
%          #/ /## ##   ##   ## ##    ##    ##            ##
%           #/ ## ##   ##   ## ##    ##    ##            ##
%            # /  ##   ##   ## ##    ##    ##            ##
%  /##        /   ##   ##   ## ##    /#    ##            ##
% /  ########/    ###  ###  ### ####/ ##   ###           ##
%/     #####       ###  ###  ### ###   ##   ###           ##
%|
% \)
%
%
%  /###           /  /
% /  ############/ #/                              #
%/     #########   ##                             ###
%#     /  #        ##                              #
% ##  /  ##        ##
%    /  ###        ##  /##      /##       /###   ###        /###
%   ##   ##        ## / ###    / ###     / #### / ###      / #### /
%   ##   ##        ##/   ###  /   ###   ##  ###/   ##     ##  ###/
%   ##   ##        ##     ## ##    ### ####        ##    ####
%   ##   ##        ##     ## ########    ###       ##      ###
%    ##  ##        ##     ## #######       ###     ##        ###
%     ## #      /  ##     ## ##              ###   ##          ###
%      ###     /   ##     ## ####    /  /###  ##   ##     /###  ##
%       ######/    ##     ##  ######/  / #### /    ### / / #### /
%         ###       ##    ##   #####      ###/      ##/     ###/
%                         /
%                        /
%                       /
%                      /
%
% About:
% ======
%
% This template is a re-implementation of the "classicthesis" template by André
% Miede. However, it uses the "memoir" class as a basis, eliminating most
% of the external packages required by "classicthesis" and thus (hopefully)
% achieving a higher compatibility with other LaTeX packages. Large parts of
% the code have been adapted (ransacked) from André Miedes code.
%
% You can find more information about "classicthesis" at CTAN:
%
% https://www.ctan.org/pkg/classicthesis
%
% More information about the "memoir" package can be found here:
%
% https://www.ctan.org/pkg/memoir
%
%
% Authors:
% ========
%
% Jan Philip Göpfert, Andreas Stöckel
%
%
% License:
% ========
%
% This LaTeX template is published under the Creative Commons Zero license. To
% the extent possible under law, the authors have waived all copyright and
% related neighboring rights to Smart Thesis. This work is published from:
% Germany.
%


%
% EXTERNAL PACKAGES
%
\usepackage[usenames,dvipsnames,svgnames]{xcolor}
\usepackage{ragged2e} % for marginnote

%
% FONT DEFINITIONS
%

% Use the following code for XeLaTeX
\ifxetex

% Setup microtype for XeLaTeX
\usepackage[expansion=false,final]{microtype}

% Load the texgyrepagella font as main font and set the ``mathpazo'' font as
% math font
\usepackage{fontspec}
\usepackage{mathpazo}
\setmainfont%
	[%
		BoldFont       = texgyrepagella-bold.otf,
		ItalicFont     = texgyrepagella-italic.otf,
		BoldItalicFont = texgyrepagella-bolditalic.otf,
		Ligatures      = TeX
	]%
	{texgyrepagella-regular.otf}

% Define a new 'widespaced' font that is used in section headings
% or chapter titles. For backwards compatibility reasons this uses
% the old calling convention \newfontfamily\fontname[<options>{font}
% instead of the newer \newfontfamily\fontname{font}[<options>]
\newfontfamily\widespaced[%
	LetterSpace=15,%
	WordSpace={1.25, 1.25, 1.25}%
]{texgyrepagella-regular.otf}%

\DeclareRobustCommand{\spacedallcaps}[1]{%
	{\widespaced\oldstylenums{\MakeTextUppercase{#1}}}%
}%
\DeclareRobustCommand{\spacedlowsmallcaps}[1]{%
	{\widespaced\oldstylenums{\textsc{\MakeTextLowercase{#1}}}}%
}%
\fi

% Use the following code for pdfLaTex
\ifpdf

% Setup microtype for pdfLaTex
\usepackage[final,tracking=true,kerning=true,spacing=nonfrench]{microtype}


\usepackage[sc]{mathpazo}

\DeclareRobustCommand{\spacedallcaps}[1]{\textls[160]{\oldstylenums{\textsc{\MakeTextUppercase{#1}}}}}%
\DeclareRobustCommand{\spacedlowsmallcaps}[1]{\textls[80]{\oldstylenums{\textsc{\MakeTextLowercase{#1}}}}}%
\fi

% Increase linespread for Palatino
\renewcommand{\linespread}{1.05}

%
% COLOR DEFINITIONS
%

\definecolor{smartgray}{gray}{0.55}
\definecolor{smartred}{HTML}{800000}
\definecolor{smartplum}{HTML}{47294F}
\definecolor{smartblue}{HTML}{193A6B}
\definecolor{smartorange}{HTML}{AA4C00}
\definecolor{smartgreen}{HTML}{3C7705}
\definecolor{smartbutter}{HTML}{A08300}

\definecolor{tangored}{HTML}{CC0000}
\definecolor{tangoplum}{HTML}{75507B}
\definecolor{tangoblue}{HTML}{3465A4}
\definecolor{tangoorange}{HTML}{F57900}
\definecolor{tangogreen}{HTML}{73D216}
\definecolor{tangobutter}{HTML}{EDD400}

\definecolor{shadecolor}{gray}{0.95}

\colorlet{heading}{smartgray}
\colorlet{ref}{smartblue}
\colorlet{cite}{smartgreen}
\colorlet{url}{smartred}
\colorlet{caption}{black}
\colorlet{margincaption}{caption}

%
% SECTIONING STYLES (toledo)
%

% See http://tex.stackexchange.com/questions/161577/memoir-hangnum-chapter-number-in-right-margin-switch for the inspiration

\setsecnumdepth{subsection}

\makeatletter
\makechapterstyle{toledo}{%
	\chapterstyle{default}

	\renewcommand*{\chapterheadstart}{\vspace*{0.0cm}}

	\renewcommand*{\chapnumfont}{\normalfont\color{heading}\addfontfeature{SizeFeatures={Size=110}}}
	\newfont{\chapterNumber}{eurb10 scaled 7000}
	\renewcommand*{\chapnumfont}{\chapterNumber\color{heading}}
	\renewcommand*{\printchaptername}{}
	\renewcommand*{\chapternamenum}{}
	\renewcommand*{\printchapternum}{%
		\raisebox{0pt}[0pt][0pt]{%
		\makebox[0pt][l]{%
		\makebox[\dimexpr\textwidth+5.45em\relax][l]{%
			\parbox[t]{\textwidth}{\mbox{}}%
			\parbox[t]{5.45em}{\hfill\chapnumfont \thechapter}}}}}%
	\renewcommand*{\chapternamenum}{}
	\renewcommand*{\afterchapternum}{}

	\renewcommand*{\afterchaptertitle}{%
	\par\parbox{\textwidth}{\hrulefill}\par%
	\vspace*{4ex}}
	\renewcommand*{\chaptitlefont}{\normalfont\large}
	\renewcommand*{\printchaptertitle}[1]{%
		\parbox[b]{\textwidth}{%
			\raggedright\chaptitlefont\spacedallcaps{##1}%
		}%
	}

	\setsecheadstyle{\spacedlowsmallcaps}
	\setsecindent{-3.5ex}
	\setbeforesecskip{-4ex plus -1ex minus -.2ex}
	\setaftersecskip{4ex plus 1ex minus .2ex}

	\setsubsecheadstyle{\itshape}
	\setbeforesubsecskip{-4ex plus -1ex minus -.2ex}
	\setaftersubsecskip{4ex plus 1ex minus .2ex}
	\setsecnumformat{\spacedlowsmallcaps{\csname the##1\endcsname\quad}}

%	\setparaheadstyle{\normalsize\itshape}

	\renewcommand*{\abstractnamefont}{\spacedlowsmallcaps}
}
\makeatother

\chapterstyle{toledo}

%
% FIGURES
%

\captionstyle{\small}
\captionnamefont{\slshape\color{caption}}

\newsubfloat{figure}% Create subfloat in figure environment
\newsubfloat{table}% Create subfloat in table environment

%
% ACKNOWLEDGEMENT
%

\makeatletter
\newcommand{\acknowledgementname}{Acknowledgements}
\newenvironment{acknowledgement}{%
  \setup@bstract
  \if@bsrunin\else
    \ifnumber@bs \num@bs \else
      \begin{\absnamepos}\abstractnamefont\acknowledgementname\end\absnamepos%
      \vspace{\abstitleskip}%
    \fi
  \fi
  \put@bsintoc%
  \begin{@bstr@ctlist}\if@bsrunin\@bsrunintitle\fi\abstracttextfont}%
  {\par\end{@bstr@ctlist}}

\newcommand{\smartcopyright}[1]{\begingroup%
	\pagebreak%
	\thispagestyle{empty}%
	\footnotesize%
	\par\vspace*{\fill}%
	\hspace{-1.1cm}%
	\parbox{10cm}{%
		\noindent%
		Copyright \textcopyright\ \the\year\ \applytolist[, ]{}{\@author}\\[0.5cm]%
		{#1}%
	}%
\endgroup}

\newcommand{\smartinitialepigraph}[2]{\begingroup
	\cleardoublepage
	\thispagestyle{empty}
	\vspace*{2cm}
	\epigraph{#1}{#2}
	\cleardoublepage
\endgroup}

\let\oldepigraph\epigraph
\renewcommand{\epigraph}[2]{\oldepigraph{#1}{\spacedlowsmallcaps{#2}}}

\makeatother

%
% MARGINNOTE
%

%
% There are three commands for placing elements in the margin:
%
% \marginnote{<TEXT>}
% Just like \marginpar but with correct formating
%
% \marginfig{<SHORT CAPTION>}{<FIGURE CONTENT>}{<LONG CAPTION>}{\label{<LABEL>}}
% \margintbl{<SHORT CAPTION>}{<TABLE CONTENT>}{<LONG CAPTION>}{\label{<LABEL>}}
% Places a figure/table in the margin
% - <SHORT CAPTION> is the caption as it occurs in the list of figures/tables
% - <FIGURE CONTENT>/<TABLE CONTENT> is the actual figure/table content
% - <LONG CAPTION> is the caption below the figure
% - <LABEL> is the label -- we decided to let you explicitly type "\label" here
% as this allows LaTeX editors to index the label and provide auto completion.
%

% Define the "marginnote" command. For compatibility reasons
% we do not override marginpar.
\strictpagecheck
\newcommand{\marginnote}[1]{\hspace*{0pt}\marginpar{
	\microtypesetup{disable}%
	\slshape\footnotesize%
	\vspace*{-0.725em}%
	\checkoddpage%
	\ifoddpage%
		\RaggedRight%
	\else%
		\RaggedLeft%
	\fi%
	{#1}%
	\microtypesetup{enable}%
}\ignorespaces}

\newcommand{\marginfloat}[8]{\marginnote{{#1}%
\par\refstepcounter{#4}\textcolor{margincaption}{#5 #6}: {#2}#3}%
\addcontentsline{#7}{#4}{\protect\numberline {#6}{\ignorespaces #8}}\ignorespaces}

\newcommand{\marginfig}[4]{\marginfloat{#2}{#3}{#4}{figure}{Figure}{\thefigure}{lof}{#1}}
\newcommand{\margintbl}[4]{\marginfloat{#2}{#3}{#4}{table}{Table}{\thetable}{lot}{#1}}

%
% PAGE STYLE (headers, footers)
%

\makeatletter

% Deactivate capitalization of headers.
\nouppercaseheads

% Define the new "berlin" style.
\makepagestyle{berlin}

% Display the page number in the footer.
\makeevenfoot{berlin}{\thepage}{}{}
\makeoddfoot{berlin}{}{}{\thepage}

% Override the chapter style with the unfinished "berlin" style
% to eliminate the header and to make sure a consistent footer
% style is used.
\copypagestyle{chapter}{berlin}

% Define the "berlin" style
\makepsmarks{berlin}{%
	\def\chaptermark##1{%
	\markboth{\memUChead{%
		\ifnum \c@secnumdepth >\m@ne
		\if@mainmatter
			\@chapapp\ \thechapter. \ %
		\fi
		\fi
		##1}}{}}%
	\def\tocmark{\markboth{\memUChead{\contentsname}}{\memUChead{\contentsname}}}%
	\def\lofmark{\markboth{\memUChead{\listfigurename}}{\memUChead{\listfigurename}}}%
	\def\lotmark{\markboth{\memUChead{\listtablename}}{\memUChead{\listtablename}}}%
	\def\bibmark{\markboth{\memUChead{\bibname}}{\memUChead{\bibname}}}%
	\def\indexmark{\markboth{\memUChead{\indexname}}{\memUChead{\indexname}}}%
	\def\sectionmark##1{%
	\markright{\memUChead{%
		\ifnum \c@secnumdepth > \z@
		\thesection. \ %
		\fi %
		##1}}}}
\makepsmarks{berlin}{%
	\createmark{chapter}{left}{nonumber}{}{}
	\createmark{section}{right}{shownumber}{}{\ }
	\createplainmark{toc}{both}{\contentsname}
	\createplainmark{lof}{both}{\listfigurename}
	\createplainmark{lot}{both}{\listtablename}
	\createplainmark{bib}{both}{\bibname}
	\createplainmark{index}{both}{\indexname}
	\createplainmark{glossary}{both}{\glossaryname}
}

% Discriminate between "oneside" and "twoside" documents.
\if@twoside
	\makeevenhead{berlin}{\spacedlowsmallcaps{\leftmark}}{}{}
	\makeoddhead{berlin}{}{}{\spacedlowsmallcaps{\rightmark}}
\else
	\makeoddhead{berlin}{\spacedlowsmallcaps{\rightmark}}{}{}
\fi

% Use the berlin style.
\pagestyle{berlin}

\makeatother

%
% FOOTNOTES
%

\footmarkstyle{\textsc{#1}}
\setlength{\footmarkwidth}{-0.5em}
\setlength{\footmarksep}{0.5em}
\setlength{\footparindent}{0em}

%
% EPIGRAPHS
%
\makeatletter

% Make the epigraph span two third of the text width
\setlength{\epigraphwidth}{0.66\textwidth}

% Disable the rule below the epigraph
\setlength{\epigraphrule}{0pt}

% Use the normal font size
\epigraphfontsize{\normalsize}

% Style the epigraph itself -- use italic text and
% flush it to the right
\newenvironment{epigraphstyle}{
%	\microtypesetup{disable}%
	\small\slshape%
%	\RaggedLeft%
}{%
%	\microtypesetup{enable}%
}
\epigraphtextposition{epigraphstyle}

% Style the epigraph source -- prepend it with a em-dash
% and make sure no indentation is added after the epigraph
\newenvironment{epigraphsourcestyle}%
{%#
	--- \small\raggedleft\color{smartred}%
}{%
	\@afterindentfalse\@afterheading%
}
\epigraphsourceposition{epigraphsourcestyle}

\makeatother

%
% TITLEPAGE
%

\makeatletter
\newcommand\thesistype[1]{\renewcommand\@thesistype{#1}}
\newcommand\@thesistype{}

\newcommand\discipline[1]{\renewcommand\@discipline{#1}}
\newcommand\@discipline{}

\newcommand\institution[1]{\renewcommand\@institution{#1}}
\newcommand\@institution{}

\newcommand\supervisors[1]{\renewcommand\@supervisors{#1}}
\newcommand\@supervisors{}

\newcommand\internalid[1]{\renewcommand\@internalid{#1}}
\newcommand\@internalid{}

\newcommand{\applytolist}[3][{,\\\vspace{0.1cm}}]{%
	\def\nextitem{\def\nextitem{#1}}%
	\@for \el:=#3\do{\nextitem{#2{\el}}}%
}

\newcommand{\smarttitle}{
{\begingroup
\thispagestyle{empty}
\makeatletter
\raggedright
\vfill
{\LARGE \spacedallcaps{\@thesistype} \\}
{\Large \spacedlowsmallcaps{\@discipline}\\}
\vfill
{\Huge \textcolor{smartred}{\spacedallcaps{\@title}}\\}
\vfill
{\LARGE \applytolist{\spacedallcaps}{\@author}\\}
\vfill
{\Large \applytolist{\itshape}{\@institution}}
\vfill
{\Large \spacedlowsmallcaps{Supervised by}\\}
\vspace{0.1cm}
{\large \applytolist{\spacedallcaps}{\@supervisors}}
\vfill
\vfill
{\Large \spacedlowsmallcaps{\@date} \hfill \spacedlowsmallcaps{\@internalid}}
\vfill
\makeatother
\endgroup}}

\makeatletter
%
% PAGE LAYOUT
%
\settypeblocksize{23.45cm}{11.85cm}{*}
%\setlength{\textheight}{23.45cm}
%\setlength{\textwidth}{11.85cm}
\setlrmargins{3.141592653589793cm}{*}{*}
\setulmargins{*}{*}{*}
\setmarginnotes{0.8cm}{3.5cm}{1cm}
\checkandfixthelayout

\usepackage{csquotes} % Context sensitive quotation facilities. Recommended by babel and should be loaded before babel.

\usepackage[english]{babel} % Sets the language used. Essential for proper hyphenation. Translates key words like 'figure' or 'table'. Note that 'english' is American English.

\usepackage{latexsym,amsmath,amssymb,amsthm,amscd} % Provides math enviroments, math symbols and more things useful for math.

\usepackage{mathtools} % Enhances amsmath and provides further mathematical tools.

\usepackage{graphicx} % Provides inclusion of graphics and a proper interface for '\in­clude­graph­ics'.
% Su­per­sedes 'epsfig' and 'graphics'.

\usepackage{enumitem} % Provides control over list environments. Supersedes 'enumerate', which gives enumerate environment an optional argument which determines the style in which the counter is printed.

%\let\newfloat\undefined % Hack to make floatrow work with memoir.
%\usepackage{floatrow} % Handles alignment of floats (figures), with centering as default.

\usepackage{xspace} % Ugly hack that sometimes helps with otherwise missing spaces.

\usepackage[backgroundcolor=white,linecolor=smartblue,bordercolor=smartblue,textsize=footnotesize]{todonotes} % Todo notes.

% Use sans-serif font in todos to clearly distinguish it from other text
\makeatletter
\renewcommand{\todo}[2][]{\@bsphack\@todo[#1]{\sffamily{#2}}\@esphack\ignorespaces}
\makeatother

\usepackage[style=alphabetic,backend=biber]{biblatex} % Bibliography.

%\usepackage{algorithm} % Algorithms.
% Use default Memoir floats for algorithms
\newcommand{\algorithmname}{Algorithm}
\newcommand{\listalgorithmname}{List of Algorithms}
\newlistof{listofalgorithms}{loa}{\listalgorithmname}
\newfloat[chapter]{algorithm}{loa}{\algorithmname}
\newfixedcaption{\falgcaption}{algorithm}
\newlistentry[chapter]{algorithm}{loa}{0}
\cftsetindents{algorithm}{0em}{2.3em}
\makeatletter
\g@addto@macro\insertchapterspace{\addtocontents{loa}{\protect\addvspace{10pt}}}
\makeatother

\usepackage{algpseudocode} % Algorithms.

\usepackage{varioref} % Automatically locates references on other pages. Load before hyperref.

%\usepackage[hidelinks]{hyperref} % Clickable references.
\usepackage[pdfencoding=auto,pdfborderstyle={/S/U/W 1},linkbordercolor=ref,colorlinks,linkcolor=ref,citecolor=cite,urlcolor=url]{hyperref} % Clickable references.

\usepackage[all]{hypcap} % Anchors links to the beginning of their respective floats. Load after hyperref.

\usepackage[noabbrev,capitalize,nameinlink]{cleveref} % Names references automatically. Load after hyperref.

\usepackage[toc,acronym]{glossaries} % Glossary.

\usepackage{booktabs}
\usepackage{multirow}

\usepackage{pgfplots}
\pgfplotsset{compat=1.8}
\pgfplotscreateplotcyclelist{smart}{%
  tangoplum,semithick,every mark/.append style={fill=tangoplum!80!black},mark=*\\%
  tangogreen,semithick,every mark/.append style={fill=tangogreen!80!black},mark=square*\\%
  tangoorange,semithick,every mark/.append style={fill=tangoorange!80!black},mark=otimes*\\%
  tangoblue,semithick,mark=star\\%
  tangobutter,semithick,every mark/.append style={fill=tangobutter!80!black},mark=diamond*\\%
  tangored,semithick,every mark/.append style={solid,fill=tangored!80!black},mark=*\\%
}
\pgfplotsset{cycle list name=smart}


% Abbreviations
\newcommand*{\eg}{e.\,g.\@\xspace}
\newcommand*{\ie}{i.\,e.\@\xspace}
\newcommand*{\cf}{cf.\@\xspace}
\newcommand*{\etal}{et.\@ al.\@\xspace}
%\newcommand*{\etc}{etc.\@\xspace}
\makeatletter
\newcommand*{\etc}{%
  \@ifnextchar{.}%
    {etc}%
      {etc.\@\xspace}%
}
\makeatother

% Sets
\newcommand*{\R}{\mathbb{R}}
\newcommand*{\Q}{\mathbb{Q}}
\newcommand*{\Z}{\mathbb{Z}}
\newcommand*{\N}{\mathbb{N}}

% Matrix/vector operations
\newcommand*{\transpose}[1]{{#1^{\top}}}
\newcommand*{\inverse}[1]{{#1^{-1}}}
\newcommand*{\conjugate}[1]{{#1^{\ast}}}
\newcommand*{\pseudoinverse}[1]{{#1^{+}}}

% Stochastics
\newcommand*{\probability}[1]{{\mathrm{Pr}{#1}}}
\newcommand*{\expectation}[1]{{#1}}

% Optimization
\DeclareMathOperator*{\argmax}{arg\,max}
\DeclareMathOperator*{\argmin}{arg\,min}
\DeclareMathOperator{\lb}{lb}
\DeclareMathOperator{\sign}{sgn}

\usepackage{lipsum}
\usepackage[table]{xcolor}
\usepackage{minted}

\addbibresource{main.bib}

\makeglossaries
\newglossaryentry{transmittance}{
    name=Transmittance,
    description={The transmittance $T \in [0,1]$ is the fraction of light passing straight between two points}
}

\newglossaryentry{radiance}{
    name=Radiance,
    description={The radiance $L$ is the fraction of reflected light}
}

% TODO: Do I need this?
\newglossaryentry{pinhole_camera}{
    name=Pinhole Camera,
    description={Camera model where a given space is projected onto a plane through a point}
}

\newglossaryentry{albedo}{
    name=Albedo,
    description={The fraction of light a body reflects diffusely. Value in $[0, 1]$ given separately for each color channel, where $0$ denotes no reflection and $1$ denotes full reflection}
}

\newglossaryentry{simd}{
    name=SIMD,
    description={SIMD (Single Instruction Multiple Data) is a form of parallelism that performs a single operation on multiple pieces of data in parallel on a single CPU}
}


\thesistype{Bachelor Thesis}
\discipline{Computer Science}
\title{Parallel Gaussian Raytracing on the CPU}
\author{Sebastian Dawid}
\institution{Bielefeld University,Technical Faculty,Visual AI for Extended Reality Group}
\supervisors{Prof.\@~Dr.\@~Helge Rhodin,Prof.\@~Dr.-Ing.\@~Ralf M\"oller}

\newcommand*{\erf}{\text{erf}}

\makepagestyle{abs}

% Display the page number in the footer.
\makeevenfoot{abs}{\thepage}{}{}
\makeoddfoot{abs}{}{}{\thepage}

\begin{document}
    \frontmatter
    \smarttitle
    \newpage
    \tableofcontents*

    \clearpage
    \thispagestyle{abs}
    \abstractintoc
    \begin{abstract}
        \lipsum[1]
    \end{abstract}

    \mainmatter
    \chapter{Introduction}

    \section{Gaussian Ray Tracing}
    \label{sec:int_grt}
    %@TODO: propose or describe?
    In their 2015 Paper \citetitle{Rhodin:2015} \cite{Rhodin:2015} Rhodin \etal describe a volumetric image formation
    model based on a parametric density representation $D(\mathbf{x})$ given as the sum of scaled isotropic Guassians
    $\mathcal{G} = \{ G_q \}_q$. The density $D$ is then given as:
    \begin{align}
        D(\mathbf{x}) = \sum_{G_q \in \mathcal{G}} G_q(\mathbf{x})
        \label{eq:density}\\
        G_q(\mathbf{x}) = c_q \cdot \exp{\left( - \frac{\Vert\mathbf{x} - \mu_q\Vert_2^2}{2\sigma_q^2} \right)}
        \label{eq:gaussian}
    \end{align}
    where $c_q$ describes the magnitude, $\mu_q$ the center and $\sigma_q$ the standard deviation of the Gaussian $G_q$.
    Additionally an \gls{albedo} attribute $\mathbf{a}_q$ is defined for each Gaussian to denote its color. This leads
    to the scene representation $\gamma = \{ c_q, \mu_q, \sigma_q, \mathbf{a}_q \}$. The $\gamma$ is omitted from
    $G_q(\mathbf{x})$ for readability and since the parameters are given implicitly via $q$.

    To get the final color of a pixel they determine the amount of light that reaches the camera from each point
    along a ray. For this purpose they determine the \gls{transmittance} $T$ of a point at distance $s$ along a ray from
    a camera position $\mathbf{o}$ in direction $\mathbf{n}$ as:
    \begin{equation}
        T(\mathbf{o}, \mathbf{n}, s, \gamma) = \exp{\left( - \int_0^s D(\mathbf{o} + t\mathbf{n}) dt \right)}
        \label{eq:transmittance}
    \end{equation}

    For the Gaussian density representation the density along a ray $\mathbf{x} = \mathbf{o} + s\mathbf{n}$ through a
    sum of 3D Gaussians is a sum of 1D Gaussians, where the 1D Gaussians are given by inserting the ray into the
    3D Gaussians $G_q$ (\refeq{eq:gaussian}). This results in the form
    $\bar{c} \exp{\left( - \frac{(x - \bar{\mu})^2}{2\bar{\sigma}^2} \right)}$ for the 1D scaled Gaussians, with
    $\bar{\mu} = (\mu - \mathbf{o})^T\mathbf{n}$, $\bar{\sigma} = \sigma$ and
    $\bar{c} = c \cdot \exp{\left( - \frac{(\mu - \mathbf{o})^T(\mu - \mathbf{o}) - \bar{\mu}^2}{2\bar{\sigma}^2} \right)}$.

    The \gls{transmittance} can be expressed analytically using the error function
    $\erf{(x)} = \frac{2}{\sqrt{\pi}}\int_0^s \exp{(-t^2)} dt$ and gaussian form of the density, as follows:
    \begin{equation}
        \begin{aligned}
            T(\mathbf{o}, \mathbf{n}, s, \gamma) &= \exp{\left( -\int_0^s
                \sum_q G_q(\mathbf{o} + t\mathbf{n} ) dt \right)}\\
            &= \exp{\left( \sum_q \frac{\bar{\sigma}_q \bar{c}_q}{\sqrt{\frac{2}{\pi}}}
            \left( \erf{\left( \frac{-\bar{\mu}_q}{\sqrt{2}\bar{\sigma}_q} \right)}
            - \erf{\left( \frac{s - \bar{\mu}_q}{\sqrt{2}\bar{\sigma}_q} \right)} \right) \right)}\\
        \end{aligned}
        \label{eq:transmittance_analytical}
    \end{equation}

    Assuming the elements in the scene emit an equal amount of \gls{radiance}, a ray is shot through each pixel of a virtual
    \gls{pinhole_camera}. The \gls{radiance} can be computed as the product of \gls{transmittance} $T$ (\refeq{eq:transmittance}),
    density $D$ (\refeq{eq:density}), \gls{albedo} $\mathbf{a}$ and the ambient \gls{radiance} $L_e$ integrated along a ray
    $\mathbf{x} = \mathbf{o} + s\mathbf{n}$. They assume the ambient \gls{radiance} is fixed as $L_e = 1$. As such they
    disregard it in the following equation:
    \begin{equation}
        L(\mathbf{o}, \mathbf{n}, \gamma) = \int_0^\infty T(\mathbf{o}, \mathbf{n}, s, \gamma)
            \sum_q G_q(\mathbf{o} + s\mathbf{n})\mathbf{a}_q ds
    \end{equation}
    
    This integral may be approximated with sufficient accuracy by sampling around the mean of each Gaussian $G_q$
    a compact interval $S_q = \{ \bar{\mu}_q + k\lambda_q | k \in K \subset \Z \}$. For their purposes it was sufficient
    to choose $\lambda_q \sim \bar{\sigma}_q$ as the step length:
    \begin{equation}
        \hat{L}(\mathbf{o}, \mathbf{n}, \gamma) = \sum_q \mathbf{a}_q \sum_{s \in S_q}
            \lambda_q T(\mathbf{o}, \mathbf{n}, s, \gamma)G_q(\mathbf{o} + s\mathbf{n})
        \label{eq:radiance}
    \end{equation}

    Rhodin \etal describe that local sampling with $\lambda_q = \bar{\sigma}_q$ and
    $K = \{ -4, -3, \dots, 0 \}$ delivers a good enough approximation.

    \section{Gaussian Splatting}
    %@TODO: describe image formation model (splatting) as laid out in the following pagpers
    \cite{kerbl3Dgaussians}
    \cite{volume_splatting}

    \section{Ray Tracing vs. Splatting}
    %@TODO: What differentiates the two methods? Advantages? Disadvantages?

    \chapter{Optimizations}
    %@TODO: Write introductory paragraph on optimizations
    \section{SIMD}
    %@TODO: Write short introduction on SIMD
    \subsection{Notation}
    To properly rewrite the method as it is laid out in \cite{Rhodin:2015} it is
    necessary to introduce some notation to work with \gls{simd} vectors. This is
    defined in Table~\ref{tab:notation}:
    \begin{table}[h]
        \centering
        \rowcolors{1}{white}{lightgray}
        \begin{tabular}{|c|c|}
            \hline
            Notation & Definition \\
            \hline
            $x^W$ & \gls{simd} vector containing $W$ elements\\
            $x^W_i$ & $i$-th element of \gls{simd} vector $x$\\
            $\mathbf{x}^W$ & \gls{simd} vectors containing $W$ vectors $\mathbf{x} \in \R^n$\\
            $\langle \mathbf{x}^W, \mathbf{y}^W \rangle$ & elementwise inner product of the vectors in $\mathbf{x}$ and $\mathbf{y}$\\
            $[ x ]^W$ & $x$ broadcast to a \gls{simd} vector of $W$ elements\\
            $\odot$ & elementwise multiplication\\
            $\frac{x^W}{y^W}$ & elementwise division\\
            $f^W$ & function that produces a \gls{simd} vector containing $W$ elements\\\hline
        \end{tabular}
        \caption{\gls{simd} Notation}
        \label{tab:notation}
    \end{table}

    \subsection{Approaches}
    \paragraph{Parallel \gls{transmittance}:}
    \begin{equation}
        \begin{aligned}
            T^W(\mathbf{o}, \mathbf{n}, s, \gamma) &= \exp^W\left( \sum_{m = 0}^{\left\lceil \frac{|\mathcal{G}|}{W} \right\rceil - 1}
            \begin{pmatrix}
                \frac{\bar{\sigma}_{mW}\bar{c}_{mW}}{\sqrt{\frac{2}{\pi}}} \\ \vdots \\\frac{\bar{\sigma}_{(m+1)W-1}\bar{c}_{(m+1)W-1}}{\sqrt{\frac{2}{\pi}}} 
            \end{pmatrix} \right.\\&\left.\odot \begin{pmatrix}
                \erf{\left( \frac{- \bar{\mu}_{mW}}{\sqrt{2}\bar{\sigma}_{mW}} \right) - \erf{\left( \frac{s - \bar{\mu}_{mW}}{\sqrt{2}\bar{\sigma}_{mW}} \right)}} \\
                \vdots \\
               \erf{\left( \frac{- \bar{\mu}_{(m+1)W - 1}}{\sqrt{2}\bar{\sigma}_{(m+1)W - 1}} \right) - \erf{\left( \frac{s - \bar{\mu}_{(m+1)W - 1}}{\sqrt{2}\bar{\sigma}_{(m+1)W - 1}} \right)}} 
            \end{pmatrix}\right)
        \end{aligned}
    \end{equation}
    \begin{equation}
        \hat{L}(\mathbf{o}, \mathbf{n}, \gamma) = \sum_{q} \mathbf{a}_q \sum_{s \in S_q} \lambda_qG_q(\mathbf{o} + s\mathbf{n})\sum_{i = 1}^W T^W_i(\mathbf{o}, \mathbf{n}, s, \gamma)
    \end{equation}

    \paragraph{Parallel \gls{radiance}:}
    The analytical solution to the \gls{transmittance} integral from Eq.~\refeq{eq:transmittance_analytical} can be broadcast
    to operate on \gls{simd} vectors of points on rays parameterized by origins $\mathbf{o}^W$, directions $\mathbf{n}^W$
    and distances $s^W$, as follows:
    \begin{equation}
        \begin{aligned}
            T^W(\mathbf{o}^W, \mathbf{n}^W, s^W, \gamma) = \exp^W\Bigg(& \sum_q \frac{(\bar{\sigma}_q)^W
            (\bar{c}_q)^W}{\left[ \sqrt{\frac{2}{\pi}} \right]^W} \\
            \odot \Bigg(& \erf^W{\left( \frac{-(\bar{\mu}_q)^W}{[ \sqrt{2} ]^W \odot (\bar{\sigma}_q)^W} \right)}\\
            &- \erf^W{\left( \frac{s^W - (\bar{\mu}_q)^W}{[ \sqrt{2} ]^W \odot (\bar{\sigma}_q)^W} \right)} \Bigg) \Bigg) 
        \end{aligned}
        \label{eq:transmittance_broadcast}
    \end{equation}
    with
    \begin{align*}
        \bar{\mu}^W &= \left\langle [ \mu ]^W - \mathbf{o}, \mathbf{n}^W \right\rangle, \bar{\sigma}^W = \left[ \sigma \right]^W\\
        \bar{c}^W &= [c]^W \odot \exp^W{\left( - \frac{\left\langle [\mu]^W - \mathbf{o}^W, [\mu]^W - \mathbf{o}^W \right\rangle
    - \left(\bar{\mu}^W\right)^2}{[2]^W \odot \left(\bar{\sigma}^W\right)^2} \right)}
    \end{align*}

    This version of the \gls{transmittance} equation (\refeq{eq:transmittance_analytical}) allows us to rewrite the \gls{radiance} equation to operate on
    multiple gaussians at the same time as follows:
    \begin{equation}
        \begin{aligned}
            \hat{L}^W(\mathbf{o}, \mathbf{n}, \gamma) &= \sum_{m = 0}^{\left\lceil \frac{|\mathcal{G}|}{W} \right\rceil - 1} \left( \begin{pmatrix}
                \mathbf{a}_{mW}\\ \vdots \\ \mathbf{a}_{(m+1)W - 1}
            \end{pmatrix} \right.\\
            &\odot \sum_{s^W \in S_m} \left( T^W([\mathbf{o}]^W, [\mathbf{n}]^W, s^w, \gamma)\right.\\
            &\odot \left.\left.\begin{pmatrix}
                G_{mW}(\mathbf{o} + s^W_1\mathbf{n})\\ \vdots\\ G_{(m+1)W - 1}(\mathbf{o} + s^W_W\mathbf{n})
            \end{pmatrix}\right)\right)
        \end{aligned}
        \label{eq:radiance_parallel_gaussians}
    \end{equation}
    with
    \[ S_m = \left\{\left. \begin{pmatrix}
        \bar{\mu}_{mW}\\ \vdots\\ \bar{\mu}_{(m+1)W - 1}
    \end{pmatrix} + [k]^W \odot \begin{pmatrix}
        \lambda_{mW}\\ \vdots\\ \lambda_{(m+1)W - 1}
    \end{pmatrix} \,\right|\, k \in K \subset \Z \right\} \]
    and $\bar{\mu}$, $\bar{\sigma}$ and $\bar{c}$ defined the same as in Section~\ref{sec:int_grt}.

    Finally we can collect these results into the final \gls{radiance}:
    \begin{equation}
        \hat{L}(\mathbf{o}, \mathbf{n}, \gamma) = \sum_{i = 1}^W \hat{L}^W_i(\mathbf{o}, \mathbf{n}, \gamma)
        \label{eq:radiance_parallel_final}
    \end{equation}

    \paragraph{Parallel Pixels:}
    Given the broadcast version of the \gls{transmittance} equation (\refeq{eq:transmittance_broadcast}) from before I can broadcast the \gls{radiance} equation to
    calculate the \gls{radiance} for multiple pixels at once:
    \begin{equation}
        \begin{aligned}
            \hat{L}^W(\mathbf{o}^W, \mathbf{n}^W, \gamma) &= \sum_q [ \mathbf{a}_q ]^W \sum_{s \in S_q} \Big(
            [ \lambda_q ]^W \odot T^W(\mathbf{o}^W, \mathbf{n}^W, [ s ]^W, \gamma)\\
            &\odot G_q^W(\mathbf{o}^W + [ s ]^W \odot \mathbf{n}^W) \Big)
        \end{aligned}
        \label{eq:radiance_parallel_pixels}
    \end{equation}

    Note that both the \gls{transmittance} and \gls{radiance} equations still operate on the Gaussians one by one as the
    sums iterate over every Gaussian $G_q \in \mathcal{G}$ individually.

    \section{Tiling}
    %@TODO: describe the tiling method used. note similarities to splatting.
    
    \chapter{Approximations}
    %@TODO: Explain why it is necessary to approximate functions / implement own approximations
    \section{Exponential Function}
    \subsection{Cubic Spline Interpolation}
    \subsection{Bit Hack}
    \cite{fast_exp}
    
    \section{Error Function}
    \subsection{Cubic Spline Interpolation}
    \subsection{Polynomial Approximation}
    \cite{AbraSteg72}
    
    \chapter{Implementation}
    %@TODO: details TBD
    The source code of the implementation is available at: \href{https://github.com/Sebastian-Dawid/simd-gaussian-ray-tracing}{https://github.com/Sebastian-Dawid/simd-gaussian-ray-tracing}

    To enable cross compilation for different \gls{simd} extentions, here AVX2 and AVX512, I use the T-SIMD library described in \enquote{\citetitle{own_moeller_16_2}} \cite{own_moeller_16_2} by \citeauthor{own_moeller_16_2}.
    
    \chapter{Experiments}
    \section{Approximations}
    \section{Transmittance}
    \section{Full Render}
    Aside from testing singular functions or sections of the program, I have also performed a number of end-to-end tests
    to evaluate the performance improvements of the implemented optimizations in a, close to, real world scenario.
    The tests will be performed on two separate models and evaluate different aspects of the implementation,
    which I will describe in the following sections:
    \subsection{Utah/Newell Teapot}
    \href{https://graphics.stanford.edu/courses/cs148-10-summer/as3/code/as3/teapot.obj}{Model}\footnote{https://graphics.stanford.edu/courses/cs148-10-summer/as3/code/as3/teapot.obj}
    \subsection{Dense Cube}
    
    \chapter{Results and Discussion}
    \cite{hpc_toolkit}
    \section{HPCTOOLKIT}
    The derived metrics used were vector instruction waste and vector instruction efficiency defined as:
    \begin{align}
        \text{waste}(\text{cycles}, \text{instructions}) = 2 \cdot \text{cycles} - \text{instructions} \label{eq:vec_waste}\\
        \text{efficiency}(\text{cycles}, \text{instructions}) = 100 \cdot \frac{\text{instructions}}{2\cdot\text{cycles}} \label{eq:vec_efficiency}
    \end{align}
    based on the exclusive\footnote{An exclusive metric refers to the quantity of the metric measured for that scope
    alone, disregarding nested scopes such as function calls. Reference: HPCTOOLKIT User Manual page 18
    \href{https://hpctoolkit.org/manual/HPCToolkit-users-manual.pdf}{https://hpctoolkit.org/manual/HPCToolkit-users-manual.pdf}}
    \mintinline{c}{PAPI_TOT_CYC} (total cycles) and \mintinline{c}{PAPI_VEC_INS}
    (vector instructions) metrics for the given scopes.

    \chapter{Conclusion and Future Work}
    %@TODO: Short recap + what can still be done: anisotropic gaussians, loading models learned via gaussian splatting, more efficient tiling method using SIMD, performance left on the table TBD (see results)

    \appendix
    \chapter{Tables}
    \chapter{Figures}
    \backmatter
    \printglossaries
    \printbibliography[heading=bibintoc]
\end{document}
